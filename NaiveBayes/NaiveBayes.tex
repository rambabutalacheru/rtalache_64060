% Options for packages loaded elsewhere
\PassOptionsToPackage{unicode}{hyperref}
\PassOptionsToPackage{hyphens}{url}
%
\documentclass[
]{article}
\usepackage{amsmath,amssymb}
\usepackage{lmodern}
\usepackage{ifxetex,ifluatex}
\ifnum 0\ifxetex 1\fi\ifluatex 1\fi=0 % if pdftex
  \usepackage[T1]{fontenc}
  \usepackage[utf8]{inputenc}
  \usepackage{textcomp} % provide euro and other symbols
\else % if luatex or xetex
  \usepackage{unicode-math}
  \defaultfontfeatures{Scale=MatchLowercase}
  \defaultfontfeatures[\rmfamily]{Ligatures=TeX,Scale=1}
\fi
% Use upquote if available, for straight quotes in verbatim environments
\IfFileExists{upquote.sty}{\usepackage{upquote}}{}
\IfFileExists{microtype.sty}{% use microtype if available
  \usepackage[]{microtype}
  \UseMicrotypeSet[protrusion]{basicmath} % disable protrusion for tt fonts
}{}
\makeatletter
\@ifundefined{KOMAClassName}{% if non-KOMA class
  \IfFileExists{parskip.sty}{%
    \usepackage{parskip}
  }{% else
    \setlength{\parindent}{0pt}
    \setlength{\parskip}{6pt plus 2pt minus 1pt}}
}{% if KOMA class
  \KOMAoptions{parskip=half}}
\makeatother
\usepackage{xcolor}
\IfFileExists{xurl.sty}{\usepackage{xurl}}{} % add URL line breaks if available
\IfFileExists{bookmark.sty}{\usepackage{bookmark}}{\usepackage{hyperref}}
\hypersetup{
  pdftitle={Assignment3\_NaiveBayes},
  pdfauthor={Ram},
  hidelinks,
  pdfcreator={LaTeX via pandoc}}
\urlstyle{same} % disable monospaced font for URLs
\usepackage[margin=1in]{geometry}
\usepackage{color}
\usepackage{fancyvrb}
\newcommand{\VerbBar}{|}
\newcommand{\VERB}{\Verb[commandchars=\\\{\}]}
\DefineVerbatimEnvironment{Highlighting}{Verbatim}{commandchars=\\\{\}}
% Add ',fontsize=\small' for more characters per line
\usepackage{framed}
\definecolor{shadecolor}{RGB}{248,248,248}
\newenvironment{Shaded}{\begin{snugshade}}{\end{snugshade}}
\newcommand{\AlertTok}[1]{\textcolor[rgb]{0.94,0.16,0.16}{#1}}
\newcommand{\AnnotationTok}[1]{\textcolor[rgb]{0.56,0.35,0.01}{\textbf{\textit{#1}}}}
\newcommand{\AttributeTok}[1]{\textcolor[rgb]{0.77,0.63,0.00}{#1}}
\newcommand{\BaseNTok}[1]{\textcolor[rgb]{0.00,0.00,0.81}{#1}}
\newcommand{\BuiltInTok}[1]{#1}
\newcommand{\CharTok}[1]{\textcolor[rgb]{0.31,0.60,0.02}{#1}}
\newcommand{\CommentTok}[1]{\textcolor[rgb]{0.56,0.35,0.01}{\textit{#1}}}
\newcommand{\CommentVarTok}[1]{\textcolor[rgb]{0.56,0.35,0.01}{\textbf{\textit{#1}}}}
\newcommand{\ConstantTok}[1]{\textcolor[rgb]{0.00,0.00,0.00}{#1}}
\newcommand{\ControlFlowTok}[1]{\textcolor[rgb]{0.13,0.29,0.53}{\textbf{#1}}}
\newcommand{\DataTypeTok}[1]{\textcolor[rgb]{0.13,0.29,0.53}{#1}}
\newcommand{\DecValTok}[1]{\textcolor[rgb]{0.00,0.00,0.81}{#1}}
\newcommand{\DocumentationTok}[1]{\textcolor[rgb]{0.56,0.35,0.01}{\textbf{\textit{#1}}}}
\newcommand{\ErrorTok}[1]{\textcolor[rgb]{0.64,0.00,0.00}{\textbf{#1}}}
\newcommand{\ExtensionTok}[1]{#1}
\newcommand{\FloatTok}[1]{\textcolor[rgb]{0.00,0.00,0.81}{#1}}
\newcommand{\FunctionTok}[1]{\textcolor[rgb]{0.00,0.00,0.00}{#1}}
\newcommand{\ImportTok}[1]{#1}
\newcommand{\InformationTok}[1]{\textcolor[rgb]{0.56,0.35,0.01}{\textbf{\textit{#1}}}}
\newcommand{\KeywordTok}[1]{\textcolor[rgb]{0.13,0.29,0.53}{\textbf{#1}}}
\newcommand{\NormalTok}[1]{#1}
\newcommand{\OperatorTok}[1]{\textcolor[rgb]{0.81,0.36,0.00}{\textbf{#1}}}
\newcommand{\OtherTok}[1]{\textcolor[rgb]{0.56,0.35,0.01}{#1}}
\newcommand{\PreprocessorTok}[1]{\textcolor[rgb]{0.56,0.35,0.01}{\textit{#1}}}
\newcommand{\RegionMarkerTok}[1]{#1}
\newcommand{\SpecialCharTok}[1]{\textcolor[rgb]{0.00,0.00,0.00}{#1}}
\newcommand{\SpecialStringTok}[1]{\textcolor[rgb]{0.31,0.60,0.02}{#1}}
\newcommand{\StringTok}[1]{\textcolor[rgb]{0.31,0.60,0.02}{#1}}
\newcommand{\VariableTok}[1]{\textcolor[rgb]{0.00,0.00,0.00}{#1}}
\newcommand{\VerbatimStringTok}[1]{\textcolor[rgb]{0.31,0.60,0.02}{#1}}
\newcommand{\WarningTok}[1]{\textcolor[rgb]{0.56,0.35,0.01}{\textbf{\textit{#1}}}}
\usepackage{graphicx}
\makeatletter
\def\maxwidth{\ifdim\Gin@nat@width>\linewidth\linewidth\else\Gin@nat@width\fi}
\def\maxheight{\ifdim\Gin@nat@height>\textheight\textheight\else\Gin@nat@height\fi}
\makeatother
% Scale images if necessary, so that they will not overflow the page
% margins by default, and it is still possible to overwrite the defaults
% using explicit options in \includegraphics[width, height, ...]{}
\setkeys{Gin}{width=\maxwidth,height=\maxheight,keepaspectratio}
% Set default figure placement to htbp
\makeatletter
\def\fps@figure{htbp}
\makeatother
\setlength{\emergencystretch}{3em} % prevent overfull lines
\providecommand{\tightlist}{%
  \setlength{\itemsep}{0pt}\setlength{\parskip}{0pt}}
\setcounter{secnumdepth}{-\maxdimen} % remove section numbering
\ifluatex
  \usepackage{selnolig}  % disable illegal ligatures
\fi

\title{Assignment3\_NaiveBayes}
\author{Ram}
\date{10/18/2021}

\begin{document}
\maketitle

loading all the required libraries

\begin{Shaded}
\begin{Highlighting}[]
\FunctionTok{library}\NormalTok{(reshape2)}
\FunctionTok{library}\NormalTok{(gmodels)}
\FunctionTok{library}\NormalTok{(caret)}
\end{Highlighting}
\end{Shaded}

\begin{verbatim}
## Loading required package: ggplot2
\end{verbatim}

\begin{verbatim}
## Loading required package: lattice
\end{verbatim}

\begin{Shaded}
\begin{Highlighting}[]
\FunctionTok{library}\NormalTok{(ISLR)}
\FunctionTok{library}\NormalTok{(e1071)}
\FunctionTok{library}\NormalTok{(ggplot2)}
\FunctionTok{library}\NormalTok{(lattice)}
\end{Highlighting}
\end{Shaded}

Reading the Universal bank CSV file to UB

\begin{Shaded}
\begin{Highlighting}[]
\NormalTok{UB}\OtherTok{\textless{}{-}}\FunctionTok{read.csv}\NormalTok{(}\StringTok{"C:/Users/ramne/Desktop/ML Assignment/Naive Bayes/UniversalBank.csv"}\NormalTok{)}
\end{Highlighting}
\end{Shaded}

changing the numerical variables to categorical variables

\begin{Shaded}
\begin{Highlighting}[]
\NormalTok{UB}\SpecialCharTok{$}\NormalTok{Personal.Loan}\OtherTok{\textless{}{-}}\FunctionTok{factor}\NormalTok{(UB}\SpecialCharTok{$}\NormalTok{Personal.Loan)}
\NormalTok{UB}\SpecialCharTok{$}\NormalTok{Online}\OtherTok{\textless{}{-}}\FunctionTok{factor}\NormalTok{(UB}\SpecialCharTok{$}\NormalTok{Online)}
\NormalTok{UB}\SpecialCharTok{$}\NormalTok{CreditCard}\OtherTok{\textless{}{-}}\FunctionTok{factor}\NormalTok{(UB}\SpecialCharTok{$}\NormalTok{CreditCard)}
\end{Highlighting}
\end{Shaded}

Split dataset into training (60\%) and validation (40\%)

\begin{Shaded}
\begin{Highlighting}[]
\NormalTok{UBankNew }\OtherTok{\textless{}{-}} \FunctionTok{sample}\NormalTok{(}\DecValTok{2}\NormalTok{,}\FunctionTok{nrow}\NormalTok{(UB), }\AttributeTok{replace=}\ConstantTok{TRUE}\NormalTok{, }\AttributeTok{prob=}\FunctionTok{c}\NormalTok{(}\FloatTok{0.6}\NormalTok{,}\FloatTok{0.4}\NormalTok{))}
\NormalTok{tdata }\OtherTok{\textless{}{-}}\NormalTok{ UB[UBankNew}\SpecialCharTok{==}\DecValTok{1}\NormalTok{,]}
\NormalTok{vdata }\OtherTok{\textless{}{-}}\NormalTok{ UB[UBankNew}\SpecialCharTok{==}\DecValTok{2}\NormalTok{,]}
\end{Highlighting}
\end{Shaded}

A)Create a pivot table for the training data with Online as a column
variable, CC as a row variable and Loan as a secondary row variable such
that the values inside the table convey the count.

\begin{Shaded}
\begin{Highlighting}[]
\NormalTok{melt\_tbank}\OtherTok{\textless{}{-}} \FunctionTok{melt}\NormalTok{(tdata,}\AttributeTok{id=}\FunctionTok{c}\NormalTok{(}\StringTok{"CreditCard"}\NormalTok{,}\StringTok{"Personal.Loan"}\NormalTok{),}\AttributeTok{variable=}\StringTok{"Online"}\NormalTok{)}
\end{Highlighting}
\end{Shaded}

\begin{verbatim}
## Warning: attributes are not identical across measure variables; they will be
## dropped
\end{verbatim}

\begin{Shaded}
\begin{Highlighting}[]
\NormalTok{cast\_tbank}\OtherTok{\textless{}{-}}\FunctionTok{dcast}\NormalTok{(melt\_tbank,CreditCard}\SpecialCharTok{+}\NormalTok{Personal.Loan}\SpecialCharTok{\textasciitilde{}}\NormalTok{Online)}
\end{Highlighting}
\end{Shaded}

\begin{verbatim}
## Aggregation function missing: defaulting to length
\end{verbatim}

\begin{Shaded}
\begin{Highlighting}[]
\NormalTok{cast\_tbank[,}\FunctionTok{c}\NormalTok{(}\DecValTok{1}\SpecialCharTok{:}\DecValTok{2}\NormalTok{,}\DecValTok{14}\NormalTok{)]}
\end{Highlighting}
\end{Shaded}

\begin{verbatim}
##   CreditCard Personal.Loan Online
## 1          0             0   1900
## 2          0             1    205
## 3          1             0    827
## 4          1             1     88
\end{verbatim}

B)Consider the task of classifying a customer who owns a bank credit
card and is actively using online banking services. Looking at the pivot
table, what is the probability that this customer will accept the loan
offer? {[}This is the probability of loan acceptance (Loan = 1)
conditional on having a bank credit card (CC = 1) and being an active
user of online banking services (Online = 1){]}.

creating a 3 way cross table in R with the help of table function.

\begin{Shaded}
\begin{Highlighting}[]
\NormalTok{a }\OtherTok{=} \FunctionTok{table}\NormalTok{(tdata[,}\FunctionTok{c}\NormalTok{(}\DecValTok{10}\NormalTok{,}\DecValTok{13}\NormalTok{,}\DecValTok{14}\NormalTok{)])}
\NormalTok{b}\OtherTok{\textless{}{-}} \FunctionTok{as.data.frame}\NormalTok{(a)}
\NormalTok{b}
\end{Highlighting}
\end{Shaded}

\begin{verbatim}
##   Personal.Loan Online CreditCard Freq
## 1             0      0          0  791
## 2             1      0          0   83
## 3             0      1          0 1109
## 4             1      1          0  122
## 5             0      0          1  311
## 6             1      0          1   38
## 7             0      1          1  516
## 8             1      1          1   50
\end{verbatim}

It gives the cross tabulation of Online*CreditCard for
Personal.Loan=0,Personal.Loan=1 separately as shown above. There are 558
records where online = 1 and cc = 1 considering Loan=0(51) +
Loan=1(507). 52 of them accept the loan, so the conditional probability
is 52/546 = 0.09139.

C)Create two separate pivot tables for the training data. One will have
Loan (rows) as a function of Online (columns) and the other will have
Loan (rows) as a function of CC

Pivot table for Loan (rows) as a function of Online (columns)

\begin{Shaded}
\begin{Highlighting}[]
\FunctionTok{table}\NormalTok{(tdata[,}\FunctionTok{c}\NormalTok{(}\DecValTok{10}\NormalTok{,}\DecValTok{13}\NormalTok{)])}
\end{Highlighting}
\end{Shaded}

\begin{verbatim}
##              Online
## Personal.Loan    0    1
##             0 1102 1625
##             1  121  172
\end{verbatim}

Pivot table for Loan (rows) as a function of CC

\begin{Shaded}
\begin{Highlighting}[]
\FunctionTok{table}\NormalTok{(tdata[,}\FunctionTok{c}\NormalTok{(}\DecValTok{10}\NormalTok{,}\DecValTok{14}\NormalTok{)])}
\end{Highlighting}
\end{Shaded}

\begin{verbatim}
##              CreditCard
## Personal.Loan    0    1
##             0 1900  827
##             1  205   88
\end{verbatim}

D)Compute the following quantities {[}P(A \textbar{} B) means ``the
probability of A given B''{]}:

\begin{enumerate}
\def\labelenumi{\roman{enumi}.}
\tightlist
\item
  P(CC = 1 \textbar{} Loan = 1) (the proportion of credit card holders
  among the loan acceptors)
\end{enumerate}

\begin{Shaded}
\begin{Highlighting}[]
\NormalTok{P1}\OtherTok{\textless{}{-}}\FunctionTok{table}\NormalTok{(tdata[,}\FunctionTok{c}\NormalTok{(}\DecValTok{14}\NormalTok{,}\DecValTok{10}\NormalTok{)])}
\NormalTok{S1}\OtherTok{\textless{}{-}}\NormalTok{P1[}\DecValTok{2}\NormalTok{,}\DecValTok{2}\NormalTok{]}\SpecialCharTok{/}\NormalTok{(P1[}\DecValTok{2}\NormalTok{,}\DecValTok{2}\NormalTok{]}\SpecialCharTok{+}\NormalTok{P1[}\DecValTok{1}\NormalTok{,}\DecValTok{2}\NormalTok{])}
\NormalTok{S1}
\end{Highlighting}
\end{Shaded}

\begin{verbatim}
## [1] 0.3003413
\end{verbatim}

\begin{enumerate}
\def\labelenumi{\roman{enumi}.}
\setcounter{enumi}{1}
\tightlist
\item
  P(Online = 1 \textbar{} Loan = 1)
\end{enumerate}

\begin{Shaded}
\begin{Highlighting}[]
\NormalTok{P2}\OtherTok{\textless{}{-}}\FunctionTok{table}\NormalTok{(tdata[,}\FunctionTok{c}\NormalTok{(}\DecValTok{13}\NormalTok{,}\DecValTok{10}\NormalTok{)])}
\NormalTok{S2}\OtherTok{\textless{}{-}}\NormalTok{P2[}\DecValTok{2}\NormalTok{,}\DecValTok{2}\NormalTok{]}\SpecialCharTok{/}\NormalTok{(P2[}\DecValTok{2}\NormalTok{,}\DecValTok{2}\NormalTok{]}\SpecialCharTok{+}\NormalTok{P2[}\DecValTok{1}\NormalTok{,}\DecValTok{2}\NormalTok{])}
\NormalTok{S2}
\end{Highlighting}
\end{Shaded}

\begin{verbatim}
## [1] 0.5870307
\end{verbatim}

\begin{enumerate}
\def\labelenumi{\roman{enumi}.}
\setcounter{enumi}{2}
\tightlist
\item
  P(Loan = 1) (the proportion of loan acceptors)
\end{enumerate}

\begin{Shaded}
\begin{Highlighting}[]
\NormalTok{P3}\OtherTok{\textless{}{-}}\FunctionTok{table}\NormalTok{(tdata[,}\DecValTok{10}\NormalTok{])}
\NormalTok{S3}\OtherTok{\textless{}{-}}\NormalTok{P3[}\DecValTok{2}\NormalTok{]}\SpecialCharTok{/}\NormalTok{(P3[}\DecValTok{2}\NormalTok{]}\SpecialCharTok{+}\NormalTok{P3[}\DecValTok{1}\NormalTok{])}
\NormalTok{S3}
\end{Highlighting}
\end{Shaded}

\begin{verbatim}
##          1 
## 0.09701987
\end{verbatim}

\begin{enumerate}
\def\labelenumi{\roman{enumi}.}
\setcounter{enumi}{3}
\tightlist
\item
  P(CC = 1 \textbar{} Loan = 0)
\end{enumerate}

\begin{Shaded}
\begin{Highlighting}[]
\NormalTok{P4}\OtherTok{\textless{}{-}}\FunctionTok{table}\NormalTok{(tdata[,}\FunctionTok{c}\NormalTok{(}\DecValTok{14}\NormalTok{,}\DecValTok{10}\NormalTok{)])}
\NormalTok{S4}\OtherTok{\textless{}{-}}\NormalTok{P4[}\DecValTok{2}\NormalTok{,}\DecValTok{1}\NormalTok{]}\SpecialCharTok{/}\NormalTok{(P4[}\DecValTok{2}\NormalTok{,}\DecValTok{1}\NormalTok{]}\SpecialCharTok{+}\NormalTok{P4[}\DecValTok{1}\NormalTok{,}\DecValTok{1}\NormalTok{])}
\NormalTok{S4}
\end{Highlighting}
\end{Shaded}

\begin{verbatim}
## [1] 0.3032637
\end{verbatim}

\begin{enumerate}
\def\labelenumi{\alph{enumi}.}
\setcounter{enumi}{21}
\tightlist
\item
  P(Online = 1 \textbar{} Loan = 0)
\end{enumerate}

\begin{Shaded}
\begin{Highlighting}[]
\NormalTok{P5}\OtherTok{\textless{}{-}}\FunctionTok{table}\NormalTok{(tdata[,}\FunctionTok{c}\NormalTok{(}\DecValTok{13}\NormalTok{,}\DecValTok{10}\NormalTok{)])}
\NormalTok{S5}\OtherTok{\textless{}{-}}\NormalTok{P5[}\DecValTok{2}\NormalTok{,}\DecValTok{1}\NormalTok{]}\SpecialCharTok{/}\NormalTok{(P5[}\DecValTok{2}\NormalTok{,}\DecValTok{1}\NormalTok{]}\SpecialCharTok{+}\NormalTok{P5[}\DecValTok{1}\NormalTok{,}\DecValTok{1}\NormalTok{])}
\NormalTok{S5}
\end{Highlighting}
\end{Shaded}

\begin{verbatim}
## [1] 0.5958929
\end{verbatim}

\begin{enumerate}
\def\labelenumi{\roman{enumi}.}
\setcounter{enumi}{5}
\tightlist
\item
  P(Loan = 0)
\end{enumerate}

\begin{Shaded}
\begin{Highlighting}[]
\NormalTok{P6}\OtherTok{\textless{}{-}}\FunctionTok{table}\NormalTok{(tdata[,}\DecValTok{10}\NormalTok{])}
\NormalTok{S6}\OtherTok{\textless{}{-}}\NormalTok{P6[}\DecValTok{1}\NormalTok{]}\SpecialCharTok{/}\NormalTok{(P6[}\DecValTok{1}\NormalTok{]}\SpecialCharTok{+}\NormalTok{P6[}\DecValTok{2}\NormalTok{])}
\NormalTok{S6}
\end{Highlighting}
\end{Shaded}

\begin{verbatim}
##         0 
## 0.9029801
\end{verbatim}

E)Use the quantities computed above to compute the naive Bayes
probability P(Loan = 1 \textbar{} CC = 1,Online = 1).

Naive\_Bayes\_Probability= (S1 * S2 * S3)/ {[}(S1 * S2 * S3)) + (S4 * S5
* S6){]} = 0.0180/ (0.0180+0.1640) = 0.09890

F)Compare this value with the one obtained from the pivot table in (B).
Which is a more accurate estimate?

The value obtained from the Pivot table is 0.09139 and the naive bayes
probability is 0.09890 which are almost similar and the value of the
pivot table is more accurate because pivot table does not assume that
probabilities are independent.

G)Which of the entries in this table are needed for computing P(Loan = 1
\textbar{} CC = 1, Online = 1)? Run naive Bayes on the data. Examine the
model output on training data, and find the entry that corresponds to
P(Loan = 1 \textbar{} CC = 1, Online = 1). Compare this to the number
you obtained in (E).

Performing Naive Bayes on the training data

\begin{Shaded}
\begin{Highlighting}[]
\FunctionTok{table}\NormalTok{(tdata[,}\FunctionTok{c}\NormalTok{(}\DecValTok{10}\NormalTok{,}\DecValTok{13}\SpecialCharTok{:}\DecValTok{14}\NormalTok{)])}
\end{Highlighting}
\end{Shaded}

\begin{verbatim}
## , , CreditCard = 0
## 
##              Online
## Personal.Loan    0    1
##             0  791 1109
##             1   83  122
## 
## , , CreditCard = 1
## 
##              Online
## Personal.Loan    0    1
##             0  311  516
##             1   38   50
\end{verbatim}

\begin{Shaded}
\begin{Highlighting}[]
\NormalTok{tdata\_naive}\OtherTok{\textless{}{-}}\NormalTok{tdata[,}\FunctionTok{c}\NormalTok{(}\DecValTok{10}\NormalTok{,}\DecValTok{13}\SpecialCharTok{:}\DecValTok{14}\NormalTok{)]}
\NormalTok{UB\_NB}\OtherTok{\textless{}{-}}\FunctionTok{naiveBayes}\NormalTok{(Personal.Loan}\SpecialCharTok{\textasciitilde{}}\NormalTok{.,}\AttributeTok{data=}\NormalTok{tdata\_naive)}
\NormalTok{UB\_NB}
\end{Highlighting}
\end{Shaded}

\begin{verbatim}
## 
## Naive Bayes Classifier for Discrete Predictors
## 
## Call:
## naiveBayes.default(x = X, y = Y, laplace = laplace)
## 
## A-priori probabilities:
## Y
##          0          1 
## 0.90298013 0.09701987 
## 
## Conditional probabilities:
##    Online
## Y           0         1
##   0 0.4041071 0.5958929
##   1 0.4129693 0.5870307
## 
##    CreditCard
## Y           0         1
##   0 0.6967363 0.3032637
##   1 0.6996587 0.3003413
\end{verbatim}

Confusion Matrix

\begin{Shaded}
\begin{Highlighting}[]
\NormalTok{c\_prediction}\OtherTok{\textless{}{-}}\FunctionTok{predict}\NormalTok{(UB\_NB,}\AttributeTok{newdata =}\NormalTok{tdata)}
\FunctionTok{confusionMatrix}\NormalTok{(c\_prediction,tdata}\SpecialCharTok{$}\NormalTok{Personal.Loan)}
\end{Highlighting}
\end{Shaded}

\begin{verbatim}
## Confusion Matrix and Statistics
## 
##           Reference
## Prediction    0    1
##          0 2727  293
##          1    0    0
##                                           
##                Accuracy : 0.903           
##                  95% CI : (0.8919, 0.9133)
##     No Information Rate : 0.903           
##     P-Value [Acc > NIR] : 0.5156          
##                                           
##                   Kappa : 0               
##                                           
##  Mcnemar's Test P-Value : <2e-16          
##                                           
##             Sensitivity : 1.000           
##             Specificity : 0.000           
##          Pos Pred Value : 0.903           
##          Neg Pred Value :   NaN           
##              Prevalence : 0.903           
##          Detection Rate : 0.903           
##    Detection Prevalence : 1.000           
##       Balanced Accuracy : 0.500           
##                                           
##        'Positive' Class : 0               
## 
\end{verbatim}

\begin{Shaded}
\begin{Highlighting}[]
\FunctionTok{CrossTable}\NormalTok{(}\AttributeTok{x=}\NormalTok{tdata}\SpecialCharTok{$}\NormalTok{Personal.Loan,}\AttributeTok{y=}\NormalTok{c\_prediction, }\AttributeTok{prop.chisq =} \ConstantTok{FALSE}\NormalTok{) }
\end{Highlighting}
\end{Shaded}

\begin{verbatim}
## 
##  
##    Cell Contents
## |-------------------------|
## |                       N |
## |         N / Table Total |
## |-------------------------|
## 
##  
## Total Observations in Table:  3020 
## 
##  
##                     | c_prediction 
## tdata$Personal.Loan |         0 | Row Total | 
## --------------------|-----------|-----------|
##                   0 |      2727 |      2727 | 
##                     |     0.903 |           | 
## --------------------|-----------|-----------|
##                   1 |       293 |       293 | 
##                     |     0.097 |           | 
## --------------------|-----------|-----------|
##        Column Total |      3020 |      3020 | 
## --------------------|-----------|-----------|
## 
## 
\end{verbatim}

After running the naive Bayes on the data,obtained value is 0.0957
whereas the value obtained from E is 0.0989 which are almost similar

\end{document}
